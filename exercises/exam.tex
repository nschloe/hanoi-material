% -----------------------------------------------------------------------------
% common header for all exercise sheets
\documentclass[
paper=a4,
parskip=half,
oneside
]{scrartcl}


\subject{Hanoi Fall School 2013 on Numerical Analysis}
\author{}
\date{}

\usepackage[USenglish]{babel}

\usepackage{graphicx}
% The official corporate design guidelines
% <http://www.pressestelle.tu-berlin.de/online_styleguide/design/typografie_ikonografie/>
% advise to set the logo flush right and at 22% of the pagewidth. Be somewhat
% more conservative with the size here.
%
% Use
\usepackage{pbox}
% to getting an authors box that is exactly as wide as the longest line in it,
% cf. <https://www.tug.org/tugboat/tb25-2/tb81law.pdf>.
\titlehead{
\pbox{\textwidth}{Prof.\ Dr.\ G\"unter B\"arwolff\\
                  Dr.\ Antoine Laurain\\
                  Dr.\ Christian Schr\"oder\\
                  Tobias Ahnert \\
                  Ute Kandler\\
                  Dr.\ Nico Schl\"omer}
\hfill
\includegraphics[height=0.1\textwidth]{logos/bms-logo-eff}
\hfill
%\includegraphics[width=0.15\paperwidth]{logos/tu-logo-long}
\includegraphics[height=0.1\textwidth]{logos/tu_logo_short}
}

% We're not providing \author or \date, so define a new \maketitle that
% ignores those (and doesn't reserve any extra space), cf.
% <http://tex.stackexchange.com/questions/134851/when-author-date-fields-blank-remove-whitespace-between-header-and-text>.
\makeatletter
\renewcommand{\@maketitle}{\null\vskip 2em
\begin{center}
  \ifx\@subject\@empty \else
    {\subject@font \@subject \par}
    \vskip 1.5em
  \fi
  \titlefont\huge \@title\par
  \vskip .5em
  {\ifx\@subtitle\@empty\else\usekomafont{subtitle}\@subtitle\par\fi}%
\end{center}
\vskip 2em}
\makeatother

% use the latin modern font
\usepackage[T1]{fontenc}
\usepackage{lmodern}

% use microtype
\usepackage{microtype}

\usepackage[
bookmarks,
colorlinks
]{hyperref}

% define MATLAB(R) symbol
\newcommand\matlab{{MATLAB\textsuperscript{\textregistered}}}

% redefine the enumerate environment in such a way that the pattern looks like
%
% (a)
% (b) (i)
%     (ii)
% (c)
%
\renewcommand\theenumi   {\alph{enumi}}
\renewcommand\labelenumi {(\theenumi)}
\renewcommand\theenumii  {\roman{enumii}}
\renewcommand\labelenumii{(\theenumii)}


\usepackage{algpseudocode}

\title{Final Examination}

\newtheorem{algorithm}{Algorithm}
% -----------------------------------------------------------------------------
\begin{document}

\maketitle

\section*{Solving linear equations}

\paragraph{Problem 1}
Consider the linear system of equations
\begin{equation*}
 \begin{bmatrix}
  10^{-20} & 1\\
  2 & 3
 \end{bmatrix}
 \begin{bmatrix}
  x_1\\x_2
 \end{bmatrix}
=\begin{bmatrix}
  \frac{1}{2}\\0
 \end{bmatrix}.
\end{equation*}
\begin{enumerate}
 \item Compute the growth factor $g(A)$ of the LU factorization without
   pivoting.
 \item Compute the condition number $\kappa_1(A)$ with respect to the
   1-norm.

 Hint: $\begin{bmatrix}
         a&b\\
         c&d
        \end{bmatrix}^{-1}=\frac{1}{ad-bc}\begin{bmatrix}
        d&-b\\
        -c&a
         \end{bmatrix} $, $\quad \|A\|_1=\max\limits_{j=1,\dots,n}\sum\limits_{i=1}^m|a_{ij}|$.
       \item Can we trust the result from the LU factorization without
         pivoting if we assume that $\varepsilon_{\text{mach}}=10^{-16}$?
         Explain why.
\end{enumerate}

\paragraph{Problem 2}
Find a system of equations $Ax=b$ with $A\in\R^{n\times n}$ nonsingular,
symmetric, and \emph{indefinite}, $b\in\R^n$, and $x_0\in\R^n$ such that the
conjugate gradient method breaks down.

Hint: Find $y\in\R^n$ and $A\in\R^{n\times n}$ such that $y^\tp Ay=0$.

\begin{algorithm}{Conjugate Gradient Iteration (CG)}\label{A:CG}\rm
\begin{algorithmic}[1]
\State
    $r_0 = b-Ax_0$,\,\, $p_1=r_0$
\For{$k=1,2,3,\ldots$ }
\State
    $\alpha_k = (r_{k-1}^\tp r_{k-1})/(p^\tp_{k}Ap_{k})$
\State
    $x_k=x_{k-1} + \alpha_kp_{k}$
\State
    $r_k=r_{k-1} - \alpha_kAp_{k}$
\State
    $\beta_k= (r_{k}^\tp r_{k})/(r_{k-1}^\tp r_{k-1})$
\State
    $p_{k+1}= r_{k}+\beta_kp_{k}$
\EndFor
\end{algorithmic}
\end{algorithm}
% -----------------------------------------------------------------------------
\paragraph{Problem 3}
Write (on paper) a MATLAB/Octave function
\begin{center}
  \verb!function [i,j] = maxelementposition(A)!
\end{center}
that returns the position ($i$ is the row index and $j$ the column index) of
the element with largest value in magnitude of $A\in\R^{n\times n}$ .
% -----------------------------------------------------------------------------
\section*{Nonlinear optimization}
% -----------------------------------------------------------------------------
\paragraph{Problem 4 (Definitions)}
Let $f:\R^n\to\R$ and $x\in\R^n$.
\begin{enumerate}
  \item When is $d\in\R^n$ called a \emph{descent direction} of $f$ at $x$? For
    $f$ continuously differentiable at $x$, give a sufficient condition for
    $d$ to be a descent direction of $f$ at $x$.

  \item For $f$ continuously differentiable, write down the \emph{Armijo
    rule}, used for determining the step size of a general descent method.
\end{enumerate}
% -----------------------------------------------------------------------------
\paragraph{Problem 5 (Conjugate gradient method)}
Let
\[
    f: \R^n \to \R,\quad f(x) = \frac{1}{2} x^\tp A x - b^\tp x,
\]
with $A\in\R^{n\times n}$, symmetric and positive definite, $b\in\R^n$.

\begin{enumerate}
  \item Give all minimizers of $f$ and explain your answer.
  \item Given a noncritical point $x_k$ and a direction $d_k$, let
    \[
      x_{k+1} \dfn x_k + \alpha_k d_k.
    \]
    Find the $\alpha_k\ge 0$ which minimizes $f(x_{k}+\alpha d_k)$.
  \item Let $g_k\dfn \nabla f(x_k)$ and
    \[
      \begin{split}
        d_0 &= -g_0,\\
        d_{k+1} &= -g_{k+1} + \sum_{i=0}^k \beta_i^{k} d_i.
      \end{split}
    \]
    Determine $\beta_i^k$ such that
    \[
      d_{k+1}^\tp A d_j = 0 \quad\forall j\in\{0,\dots,k\}.
    \]
    What is the name of this property?
\end{enumerate}
% -----------------------------------------------------------------------------
\paragraph{Problem 6 (Newton's method)}

Let
\[
  f: \R \to \R,\quad f(x) \dfn 1 - \exp(-x^2).
\]

\begin{enumerate}

\item Show that $f$ has only one critical point $x^*$ and that it is a global
  minimizer.

\item For which $x$ is $\nabla^2 f(x)$ positive definite?

\item Let $x_0\neq x^*$. Show that the sequence of Newton iterates
  $\{x_k\}$ converges to $x^*$. Compute the quotient-convergence factor
  $Q_p(x_k)$ with $p\in[1,\infty)$.  What is the Q-convergence order of
    $\{x_k\}$?

  \item Let
    \[
       g: \R^2 \to \R,\quad g(x_1, x_2) \dfn x_1^4 + x_2^2.
    \]
    Do tasks (a), (b), (c) for $g$.
 How do you explain the difference in the Q-convergence order between $f$ and $g$?
\end{enumerate}
% -----------------------------------------------------------------------------
\section*{Numerical methods for PDEs}
% -----------------------------------------------------------------------------
\paragraph{Problem 7}
There is given the problem
\begin{align*}
  -u''  + u = \ln (1-x+\epsilon),\; x\in ]0,1[, \; 0 < \epsilon \ll 1,\\
   u(0) = 1,\quad u(1) = -1.
\end{align*}
Discretize this problem using
\[   \Omega_h = \{ x_k\;|\; x_k = \sum_{j=1}^k h_j,\; k=0,\dots, n+1\} \,,\;
     \Gamma_h = \{ 0\} \cup \{ 1\}   \]
with $h_k > 0$ and $\sum_{j=1}^{n+1} h_j = 1$,
by a consistent (this is also to show) finite difference approximation.

Show that the resulting tridiagonal matrix $A = (a_{ij})_{i,j=1}^n$ is
%is strict diagonal dominant, i.e.
%\[   \sum_{j=1, j\ne i}^n  |a_{ij}| < |a_{ii}|\;,\quad   \mbox{for all}\; i=1,\dots,n\;,\]
%and further
symmetric and positive definite (i.e., $x^\tp A x > 0$ for all $x\in \R^n,\, x\ne \boldsymbol{0}$).
% -----------------------------------------------------------------------------
\paragraph{Problem 8}
For $x\in \Omega_h \subset \R$ there are available the
neighboring points
\[
  x-h,\; x+h,\; x+2h \subset \Omega_h.
\]
Construct a finite difference approximation $D_h u(x)$ of the derivative
$\frac{\partial u}{\partial x}(x)$
of third order, i.e.,
\[
(D_h u)(x) - \frac{\partial u}{\partial x}(x) = O (h^3),
\]
where we suppose $u \in C^\infty$.
% -----------------------------------------------------------------------------
\paragraph{Problem 9*}
For the unit square $\Omega = ]0,1[\times ]0,1[$ there is given
a very regular equidistant discretization into
squared Voronoi boxes
\[
\Omega_{ij} = ]i\,h,(i+1)h[\times ]j\, h,(j+1)h[,\; i,j=0,\dots,n,\; h=1/(n+1),\; n\in \N,\; n>1.
\]
There is given a continuous differentiable vector field
\[
  \vec{q}(x,y) = \binom{p(x,y)}{q(x,y)} : \R^2 \rightarrow \R^2.
\]
The discretized local balance on $\bar\Omega_{ij}$ should be
\begin{align*}
  b_{ij} & = p((i+1)h,(j+\tfrac{1}{2})h)\,h - p(i\,h,(j+\tfrac{1}{2})h)h\\
         & + q((i+\tfrac{1}{2})h,(j+1)h)h - q((i+\tfrac{1}{2})h,j\,h)h,\; i,j=0,\dots,n.
\end{align*}
Show that the sum of the local balances
\[
\sum_{i,j=0}^n b_{i,j}
\]
equals the global balance
\begin{equation}\label{gb}
   \int_{\partial \Omega} \vec{q} \cdot \vec{n}  \, \dd\partial\Omega,
\end{equation}
where $\vec{n}$ is the outward unit normal vector and where we compute the
integral \eqref{gb} by a midpoint quadrature rule, i.e.,
\[
\int_\Gamma f(x,y) \, \dd\Gamma  \approx f(x_g,y_g) \int_\Gamma \,\dd\Gamma.
\]
($(x_g,y_g)$ is the center of gravity of the surface $\Gamma$.)
% -----------------------------------------------------------------------------


\end{document}
