% -----------------------------------------------------------------------------
% common header for all exercise sheets
\documentclass[
paper=a4,
parskip=half,
oneside
]{scrartcl}


\subject{Hanoi Fall School 2013 on Numerical Analysis}
\author{}
\date{}

\usepackage[USenglish]{babel}

\usepackage{graphicx}
% The official corporate design guidelines
% <http://www.pressestelle.tu-berlin.de/online_styleguide/design/typografie_ikonografie/>
% advise to set the logo flush right and at 22% of the pagewidth. Be somewhat
% more conservative with the size here.
%
% Use
\usepackage{pbox}
% to getting an authors box that is exactly as wide as the longest line in it,
% cf. <https://www.tug.org/tugboat/tb25-2/tb81law.pdf>.
\titlehead{
\pbox{\textwidth}{Prof.\ Dr.\ G\"unter B\"arwolff\\
                  Dr.\ Antoine Laurain\\
                  Dr.\ Christian Schr\"oder\\
                  Tobias Ahnert \\
                  Ute Kandler\\
                  Dr.\ Nico Schl\"omer}
\hfill
\includegraphics[height=0.1\textwidth]{logos/bms-logo-eff}
\hfill
%\includegraphics[width=0.15\paperwidth]{logos/tu-logo-long}
\includegraphics[height=0.1\textwidth]{logos/tu_logo_short}
}

% We're not providing \author or \date, so define a new \maketitle that
% ignores those (and doesn't reserve any extra space), cf.
% <http://tex.stackexchange.com/questions/134851/when-author-date-fields-blank-remove-whitespace-between-header-and-text>.
\makeatletter
\renewcommand{\@maketitle}{\null\vskip 2em
\begin{center}
  \ifx\@subject\@empty \else
    {\subject@font \@subject \par}
    \vskip 1.5em
  \fi
  \titlefont\huge \@title\par
  \vskip .5em
  {\ifx\@subtitle\@empty\else\usekomafont{subtitle}\@subtitle\par\fi}%
\end{center}
\vskip 2em}
\makeatother

% use the latin modern font
\usepackage[T1]{fontenc}
\usepackage{lmodern}

% use microtype
\usepackage{microtype}

\usepackage[
bookmarks,
colorlinks
]{hyperref}

% define MATLAB(R) symbol
\newcommand\matlab{{MATLAB\textsuperscript{\textregistered}}}

% redefine the enumerate environment in such a way that the pattern looks like
%
% (a)
% (b) (i)
%     (ii)
% (c)
%
\renewcommand\theenumi   {\alph{enumi}}
\renewcommand\labelenumi {(\theenumi)}
\renewcommand\theenumii  {\roman{enumii}}
\renewcommand\labelenumii{(\theenumii)}


\usepackage{algpseudocode}

\title{Final Examination}

\newtheorem{algorithm}{Algorithm}
% -----------------------------------------------------------------------------
\begin{document}

\maketitle

\section*{Solving linear equations}

\paragraph{Problem 1}

\paragraph{Problem 2}
% -----------------------------------------------------------------------------
\paragraph{Problem 3}
% -----------------------------------------------------------------------------
\section*{Nonlinear optimization}
% -----------------------------------------------------------------------------
\paragraph{Problem 4 (Definitions)}
See syllabus.
% -----------------------------------------------------------------------------
\paragraph{Problem 5 (Conjugate gradient method)}
Let
\[
    f: \R^n \to \R,\quad f(x) = \frac{1}{2} x^\tp A x - b^\tp x,
\]
with $A\in\R^{n\times n}$, symmetric and positive definite, $b\in\R^n$.

\begin{enumerate}
  \item We have that
\[
\nabla f(x) = Ax - b,
\]
and since $A$ is invertible (because it is positive definite), the only
critical point of $f$ is $x^*=A^{-1}b$.

The Hessian of $f$, $\nabla^2 f(x) = A$, is obviously positive definite for
$x^*$, so the critical point is a minimum.

  \item See syllabus, page 36;
\[
\alpha_k = \frac{\|\nabla f(x_k)\|}{\|\nabla f(x_k)\|_A}.
\]

  \item See syllabus, page 41;
\[
\beta_i^k = \frac{g_{k+1}^\tp A d_i}{d_i^\tp A d_i} \quad\forall i\in\{1,\dots,k\}.
\]
  The directions $d_k$ are called \emph{conjugate} or \emph{$A$-conjugate}.
\end{enumerate}
% -----------------------------------------------------------------------------
\paragraph{Problem 6 (Newton's method)}

Let
\[
  f: \R \to \R,\quad f(x) \dfn 1 - \exp(-x^2).
\]

\begin{enumerate}

\item We have
\[
\nabla f(x) = 2x\exp(-x^2),
\]
so the only critical point is $x^*=0$.

Moreover, it clearly is $f(x)\ge 0$, and $f(0) = 0$, so it must be a global
minimizer.

\item
\[
\nabla^2 f(x) = 2\exp(-x^2) - 4x^2\exp(-x^2) = (2-4x^2)\exp(-x^2),
\]
so $\nabla^2 f(x)$ is positive for $\{x\in\R: |x|<\sqrt{1/2}\}$.

\item One Newton step is defined by
\[
x_{k+1} = x_k - [\nabla^2 f(x_k)]^{-1} \nabla f(x_k),
\]
so in this case
\begin{multline*}
x_{k+1}
= x_k - \frac{2x_k\exp(-x_k^2)}{(2-4x_k^2)\exp(-x_k^2)}
= x_k - \frac{x_k}{(1-2x_k^2)}\\
= \left[1 - \frac{1}{1-2x_k^2}\right] x_k
= \left[\frac{-2x_k^2}{1-2x_k^2}\right] x_k
= \underbrace{\left[\frac{1}{1-\frac{1}{2x_k^2}}\right]}_{=\gamma} x_k.
\end{multline*}
We ask ourselves the question when
\begin{equation}\label{eq:gamma}
1 > |\gamma| = \left|\frac{1}{1-\frac{1}{2x_k^2}}\right|
\end{equation}
since then, the iteration converges to the minimum $x^*=0$.
We separate two cases for $\gamma$.
\begin{itemize}
\item $1-\frac{1}{2x_k^2} > 0$, i.e., $x_k^2 > \frac{1}{2}$.
      In this case, inequality (\ref{eq:gamma}) is equivalent to
\[
1 < 1 - \frac{1}{2x_k^2},
\]
i.e.,
\[
0 < - \frac{1}{2x_k^2},
\]
a contradiction.

\item $1-\frac{1}{2x_k^2} < 0$, i.e., $x_k^2 < \frac{1}{2}$.
      Here, inequality (\ref{eq:gamma}) is equivalent to
\[
1 < -1 + \frac{1}{2x_k^2},
\]
i.e.,
\[
x_k^2 < \frac{1}{4}.
\]
Hence, as soon as $|x_k| < \frac{1}{2}$, the $|\gamma| < 1$ and the iteration
converges towards $x^*=0$.

We now turn to the quotient convergence factor $Q_p(\{x_k\})$. We have
\begin{multline*}
Q_p(\{x_k\})
= \limsup_{k\to\infty} \frac{\|x_{k+1}-x^*\|}{\|x_k-x^*\|^p}
= \limsup_{k\to\infty} \frac{\left|\frac{1}{1-1/(2x_k^2)} x_k\right|}{|x_k|^p}\\
= \limsup_{k\to\infty} \frac{1}{-1+1/(2x_k^2)} |x_k|^{1-p}
= \limsup_{k\to\infty} \frac{2}{1-2x_k^2} |x_k|^{3-p},
\end{multline*}
so
\[
Q_p(\{x_k\}) =
\begin{cases}
0 \quad&\text{for } 1\le p < 3\\
2 \quad&\text{for }  p = 3\\
+\infty \quad&\text{otherwise.}
\end{cases}
\]
Hence, the Q-order of convergence is 3.
\end{itemize}

  \item
\begin{itemize}
\item Clearly $g(x_1, x_2)\ge 0$, and $g(x^*)=0$, where $x^*=(0,0)^\tp$ is the only global minimizer.
\[
\nabla f(x) = ( 4x_1^3, 2x_2)^\tp.
\]

\item
\[
\nabla^2 f(x) =
\begin{pmatrix}
12 x_1^2 & 0\\
0 & 2
\end{pmatrix},
\]
which is positive definite for all $(x_1, x_2)$ except $x^*$.

\item
\begin{multline*}
x_{k+1}
= x_k - [\nabla^2 f(x_k)]^{-1} \nabla f(x_k)
= x_k -
\begin{pmatrix}
1/(12 x_1^2) & 0\\
0 & 1/2
\end{pmatrix}
\begin{pmatrix}
4 x_1^3\\
2 x_2
\end{pmatrix}\\
= x_k -
\begin{pmatrix}
1/3 x_1\\
x_2
\end{pmatrix}
=
\begin{pmatrix}
2/3 x_1\\
0
\end{pmatrix}
\end{multline*}

Consequently
\[
Q_p(\{x_k\})
= \limsup_{k\to\infty} \frac{\|x_{k+1}-x^*\|}{\|x_k-x^*\|^p}
= \limsup_{k\to\infty} \frac{|2/3 x_1|}{(x_1^2+x_2^2)^{p/2}}
\]
Since $x_2=0$ after one Newton step (see above), we can assume $x_2=0$ here.
Hence
\[
Q_p(\{x_k\})
= \limsup_{k\to\infty} \frac{|2/3 x_1|}{(|x_1|^p}
= \limsup_{k\to\infty} \frac{2}{3} |x_1|^{1-p},
\]
so
\[
Q_p(\{x_k\})
=
\begin{cases}
\tfrac{2}{3} \quad&\text{for } p=1,\\
+\infty \quad&\text{otherwise.}
\end{cases}
\]
That means that the order of convergence is 1.

The convergence order is not quadratic (as predicted by the convergence
analysis for Newton's method) since the Hessian is not positive definite in
$x^*$.

\end{itemize}
\end{enumerate}
% -----------------------------------------------------------------------------
\section*{Numerical methods for PDEs}
% -----------------------------------------------------------------------------
\paragraph{Problem 7}
% -----------------------------------------------------------------------------
\paragraph{Problem 8}
% -----------------------------------------------------------------------------
\paragraph{Problem 9*}
% -----------------------------------------------------------------------------


\end{document}
