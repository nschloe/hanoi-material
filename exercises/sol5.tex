% common header for all exercise sheets
\documentclass[
paper=a4,
parskip=half,
oneside
]{scrartcl}


\subject{Hanoi Fall School 2013 on Numerical Analysis}
\author{}
\date{}

\usepackage[USenglish]{babel}

\usepackage{graphicx}
% The official corporate design guidelines
% <http://www.pressestelle.tu-berlin.de/online_styleguide/design/typografie_ikonografie/>
% advise to set the logo flush right and at 22% of the pagewidth. Be somewhat
% more conservative with the size here.
%
% Use
\usepackage{pbox}
% to getting an authors box that is exactly as wide as the longest line in it,
% cf. <https://www.tug.org/tugboat/tb25-2/tb81law.pdf>.
\titlehead{
\pbox{\textwidth}{Prof.\ Dr.\ G\"unter B\"arwolff\\
                  Dr.\ Antoine Laurain\\
                  Dr.\ Christian Schr\"oder\\
                  Tobias Ahnert \\
                  Ute Kandler\\
                  Dr.\ Nico Schl\"omer}
\hfill
\includegraphics[height=0.1\textwidth]{logos/bms-logo-eff}
\hfill
%\includegraphics[width=0.15\paperwidth]{logos/tu-logo-long}
\includegraphics[height=0.1\textwidth]{logos/tu_logo_short}
}

% We're not providing \author or \date, so define a new \maketitle that
% ignores those (and doesn't reserve any extra space), cf.
% <http://tex.stackexchange.com/questions/134851/when-author-date-fields-blank-remove-whitespace-between-header-and-text>.
\makeatletter
\renewcommand{\@maketitle}{\null\vskip 2em
\begin{center}
  \ifx\@subject\@empty \else
    {\subject@font \@subject \par}
    \vskip 1.5em
  \fi
  \titlefont\huge \@title\par
  \vskip .5em
  {\ifx\@subtitle\@empty\else\usekomafont{subtitle}\@subtitle\par\fi}%
\end{center}
\vskip 2em}
\makeatother

% use the latin modern font
\usepackage[T1]{fontenc}
\usepackage{lmodern}

% use microtype
\usepackage{microtype}

\usepackage[
bookmarks,
colorlinks
]{hyperref}

% define MATLAB(R) symbol
\newcommand\matlab{{MATLAB\textsuperscript{\textregistered}}}

% redefine the enumerate environment in such a way that the pattern looks like
%
% (a)
% (b) (i)
%     (ii)
% (c)
%
\renewcommand\theenumi   {\alph{enumi}}
\renewcommand\labelenumi {(\theenumi)}
\renewcommand\theenumii  {\roman{enumii}}
\renewcommand\labelenumii{(\theenumii)}

% -----------------------------------------------------------------------------
\title{Solutions for exercise sheet 5}
\subtitle{Numerical Optimization: Basics}
% -----------------------------------------------------------------------------
\usepackage{amsmath}
% -----------------------------------------------------------------------------
\newcommand\tp{\ensuremath{\text{\upshape T}}}
% =============================================================================
\begin{document}
\maketitle

\paragraph{Exercise 1.1} % Nocedal 2.1
\begin{itemize}
\item \[
\nabla f(x) =
\begin{pmatrix}
-400 x(y-x^2)-2(1-x)\\
200(y-x^2)
\end{pmatrix}
\]
\item \[
\nabla^2 f(x) =
\begin{pmatrix}
-400x (y-3x^2) + 2 & -400x\\
-400x              & 200
\end{pmatrix}
\]
\item $\nabla f(x^*) = 0$, $x*=(1,1)^\tp$ is the only solution.
\item
\[
\nabla^2 f(x*) =
\begin{pmatrix}
802 & -400\\
-400 & 200
\end{pmatrix}
\]
is positive definite since $802>0$ and $\det(\nabla^2 f(x^*)) = 802\cdot200 - 400\cdot 400 > 0$ (Sylvester's criterion).
\end{itemize}


\paragraph{Exercise 1.2} % Nocedal 2.2
Show that the function
\[
  f(x) = 8x_1 + 12x_2 + x_1^2 - 2x_2^2
\]
has only one stationary point, and that it is neither a maximum nor a minimum,
but a saddle point. Sketch the contour lines of $f$.

\begin{itemize}
\item
\[
\nabla f(x) =
\begin{pmatrix}
8 + 2x\\
12 - 4y
\end{pmatrix}.
\]
The unique solution of $\nabla f(x^*) = 0$ is $x^*=(-4,3)^\tp$.
\item
\[
\nabla^2 f(x) =
\begin{pmatrix}
2 & 0\\
0 & - 4
\end{pmatrix}.
\]
The eigenvalues of $\nabla^2 f(x^*)$ are $\{2, -4\}$, so it is not positive-definite. Hence, $x^*$ is not a minimizer of $f$. With the same argument for $-f$ we see that $x^*$ neither is a minimizer of $-f$, i.e., a maximizer of $f$. Hence, $x^*$ is a saddle point.

% TODO plot
\end{itemize}


\paragraph{Exercise 1.3} % Nocedal 2.7
Let $S$ be the set of global minimizers of $f$. If $S$ only has one element,
then it is obviously a convex set. Otherwise for all $x, y \in S$ and
$\alpha\in [0, 1]$,
\[
f (\alpha x + (1 − \alpha)y) \le \alpha f(x) + (1 − \alpha)f (y)
\]
since $f$ is convex. $f(x) = f(y)$ since $x, y$ are both global minimizers.
Therefore,
\[
f (\alpha x + (1 − \alpha)y) \le \alpha f (x) + (1 − \alpha)f (x) = f (x).
\]
But since $f (x)$ is a global minimizing value, $f (x) \le f (\alpha x + (1 −
\alpha)y)$.  Therefore, $f (\alpha x + (1 − \alpha y) = f (x)$ and hence
$\alpha x + (1 − \alpha )y \in S$. Thus $S$ is a convex set.


\paragraph{Exercise 1.4} % Nocedal 2.8
Consider the function
\[
  f(x_1, x_2) = (x_1+x_2^2)^2.
\]
At the point $x=(1,0)^\tp$ we consider the search direction $p=(-1,1)^\tp$.
Show that $p$ is a descent direction and find all minimizers of the problem.

$−\nabla f$ indicates steepest descent. $p_k\cdot · (−\nabla f) = \|p_k\| \|\nabla f\| \cos\theta$. $p_k$ is a descent direction if $−\pi/2 < \theta < \pi /2$, i.e., $\cos \theta > 0$. This means that
\[
0 \stackrel{!}{<} \cos\theta = \frac{p_k\cdot (-\nabla f)}{\|p_k\|\|\nabla f\|},
\]
so we have to check if $p_k\cdot\nabla f < 0$:
\[
\nabla f(x) =
\begin{pmatrix}
2(x+y^2)\\
4y(x+y^2)
\end{pmatrix},
\quad
\nabla f((1,0)^\tp) =
\begin{pmatrix}
2\\
0
\end{pmatrix}.
\]
\[
p_k \cdot \nabla f
=
(-1, 1)^\tp \cdot (2,0)^\tp = -2 < 0.
\]
so indeed, $p$ is a descent direction at $(1,0)^\tp$.

\end{document}
% =============================================================================
