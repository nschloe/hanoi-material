% common header for all exercise sheets
\documentclass[
paper=a4,
parskip=half,
oneside
]{scrartcl}


\subject{Hanoi Fall School 2013 on Numerical Analysis}
\author{}
\date{}

\usepackage[USenglish]{babel}

\usepackage{graphicx}
% The official corporate design guidelines
% <http://www.pressestelle.tu-berlin.de/online_styleguide/design/typografie_ikonografie/>
% advise to set the logo flush right and at 22% of the pagewidth. Be somewhat
% more conservative with the size here.
%
% Use
\usepackage{pbox}
% to getting an authors box that is exactly as wide as the longest line in it,
% cf. <https://www.tug.org/tugboat/tb25-2/tb81law.pdf>.
\titlehead{
\pbox{\textwidth}{Prof.\ Dr.\ G\"unter B\"arwolff\\
                  Dr.\ Antoine Laurain\\
                  Dr.\ Christian Schr\"oder\\
                  Tobias Ahnert \\
                  Ute Kandler\\
                  Dr.\ Nico Schl\"omer}
\hfill
\includegraphics[height=0.1\textwidth]{logos/bms-logo-eff}
\hfill
%\includegraphics[width=0.15\paperwidth]{logos/tu-logo-long}
\includegraphics[height=0.1\textwidth]{logos/tu_logo_short}
}

% We're not providing \author or \date, so define a new \maketitle that
% ignores those (and doesn't reserve any extra space), cf.
% <http://tex.stackexchange.com/questions/134851/when-author-date-fields-blank-remove-whitespace-between-header-and-text>.
\makeatletter
\renewcommand{\@maketitle}{\null\vskip 2em
\begin{center}
  \ifx\@subject\@empty \else
    {\subject@font \@subject \par}
    \vskip 1.5em
  \fi
  \titlefont\huge \@title\par
  \vskip .5em
  {\ifx\@subtitle\@empty\else\usekomafont{subtitle}\@subtitle\par\fi}%
\end{center}
\vskip 2em}
\makeatother

% use the latin modern font
\usepackage[T1]{fontenc}
\usepackage{lmodern}

% use microtype
\usepackage{microtype}

\usepackage[
bookmarks,
colorlinks
]{hyperref}

% define MATLAB(R) symbol
\newcommand\matlab{{MATLAB\textsuperscript{\textregistered}}}

% redefine the enumerate environment in such a way that the pattern looks like
%
% (a)
% (b) (i)
%     (ii)
% (c)
%
\renewcommand\theenumi   {\alph{enumi}}
\renewcommand\labelenumi {(\theenumi)}
\renewcommand\theenumii  {\roman{enumii}}
\renewcommand\labelenumii{(\theenumii)}

% -----------------------------------------------------------------------------
\title{Exercise sheet 2}
\subtitle{Numerical Optimization: Convergence, Conjugate Gradient method}
% -----------------------------------------------------------------------------
\usepackage{amsmath}
% -----------------------------------------------------------------------------
\newcommand\tp{\ensuremath{\text{\upshape T}}}
% =============================================================================
\begin{document}
\maketitle

\paragraph{Exercise 2.1} % Nocedal 2.12
Show that the sequence $x_k=1/k$ is no Q-linearly convergent, although it does converge to $0$. (This is called \emph{sublinear convergence.})

\paragraph{Exercise 2.2} % Nocedal 2.13
Show that the sequence
\[
x_k = 1 + \left(\frac{1}{2}\right)^{2^k}
\]
is Q-quadratically convergent to 1.

% Alternatives:

%\paragraph{Exercise 2.2} % Nocedal 2.14
%Does the sequence $1/k!$ converge Q-superlinearly? Q-quadratically?

%\paragraph{Exercise 2.2} % Nocedal 2.15
%Consider the sequence $\{x_k\}$ defined by
%\[
%x_k =
%\begin{cases}
%(1/4)^{2^k},\quad&k\text{ even,}\\
%x_{k-1}/k,\quad&k\text{ odd.}
%\end{cases}
%\]
%Is the sequence Q-superlinearly convergent? Q-quadratically? R-quadratically?

\paragraph{Exercise 2.3}  % Nocedal 5.2
Show that if the nonzero vectors $p_0, p_1,\dots, p_k$ satisfy (??), where $A$
is symmetric and positive-definite, then these vectors are linearly
independent. (The results implies that $A$ has at most $n$ conjugate
directions.)

\paragraph{Exercise 2.4}  % Nocedal 5.4
Show that if $f(\cdot)$ is a strictly convex quadratic function, the function
$h(\sigma)=f(x_0+\sigma_0 p_0 + \dots \sigma_{k-1}p_{k-1})$ also is a strictly
convex quadratic function in $\sigma=(\sigma_0,\dots,\sigma_{k-1})^\tp$.

\paragraph{Programming 2} % Nocedal 5.1
Implement the Conjugate Gradient algorithm and use it to solve linear systems
with \emph{Hilbert matrices} $H$,
\[
H_{i,j} = \frac{1}{i+j+1}.
\]
Set the right-hand side to $b=(1,\dots,1)^\tp$ and the initial point to $x_0=(0,\dots,0)^\tp$. Try dimensions $5, 8, 12, 20$, and report the number of iterations required to reduce the residual below $10^{-6}$.

\end{document}
% =============================================================================
