% common header for all exercise sheets
\documentclass[
paper=a4,
parskip=half,
oneside
]{scrartcl}


\subject{Hanoi Fall School 2013 on Numerical Analysis}
\author{}
\date{}

\usepackage[USenglish]{babel}

\usepackage{graphicx}
% The official corporate design guidelines
% <http://www.pressestelle.tu-berlin.de/online_styleguide/design/typografie_ikonografie/>
% advise to set the logo flush right and at 22% of the pagewidth. Be somewhat
% more conservative with the size here.
%
% Use
\usepackage{pbox}
% to getting an authors box that is exactly as wide as the longest line in it,
% cf. <https://www.tug.org/tugboat/tb25-2/tb81law.pdf>.
\titlehead{
\pbox{\textwidth}{Prof.\ Dr.\ G\"unter B\"arwolff\\
                  Dr.\ Antoine Laurain\\
                  Dr.\ Christian Schr\"oder\\
                  Tobias Ahnert \\
                  Ute Kandler\\
                  Dr.\ Nico Schl\"omer}
\hfill
\includegraphics[height=0.1\textwidth]{logos/bms-logo-eff}
\hfill
%\includegraphics[width=0.15\paperwidth]{logos/tu-logo-long}
\includegraphics[height=0.1\textwidth]{logos/tu_logo_short}
}

% We're not providing \author or \date, so define a new \maketitle that
% ignores those (and doesn't reserve any extra space), cf.
% <http://tex.stackexchange.com/questions/134851/when-author-date-fields-blank-remove-whitespace-between-header-and-text>.
\makeatletter
\renewcommand{\@maketitle}{\null\vskip 2em
\begin{center}
  \ifx\@subject\@empty \else
    {\subject@font \@subject \par}
    \vskip 1.5em
  \fi
  \titlefont\huge \@title\par
  \vskip .5em
  {\ifx\@subtitle\@empty\else\usekomafont{subtitle}\@subtitle\par\fi}%
\end{center}
\vskip 2em}
\makeatother

% use the latin modern font
\usepackage[T1]{fontenc}
\usepackage{lmodern}

% use microtype
\usepackage{microtype}

\usepackage[
bookmarks,
colorlinks
]{hyperref}

% define MATLAB(R) symbol
\newcommand\matlab{{MATLAB\textsuperscript{\textregistered}}}

% redefine the enumerate environment in such a way that the pattern looks like
%
% (a)
% (b) (i)
%     (ii)
% (c)
%
\renewcommand\theenumi   {\alph{enumi}}
\renewcommand\labelenumi {(\theenumi)}
\renewcommand\theenumii  {\roman{enumii}}
\renewcommand\labelenumii{(\theenumii)}

% -----------------------------------------------------------------------------
\title{Exercise sheet 5}
\subtitle{Numerical Optimization: Basics}
% -----------------------------------------------------------------------------
\usepackage{amsmath}
% -----------------------------------------------------------------------------
\newcommand\tp{\ensuremath{\text{\upshape T}}}
% =============================================================================
\begin{document}
\maketitle

\paragraph{Exercise 1.1} % Nocedal 2.1
Compute  the gradient $\nabla f(x)$ and Hessian $\nabla^2f(x)$ of the
\emph{Rosenbrock function}
\begin{equation}\label{eq:rosenbrock}
  f(x) = 100(x_2-x_1^2)^2 + (1-x_1)^2.
\end{equation}
Show that $x^*=(1,1)^\tp$ is the only local minimizer of this function, and
that the Hessian matrix at that point is positive-definite.

\paragraph{Exercise 1.2} % Nocedal 2.2
Show that the function
\[
  f(x) = 8x_1 + 12x_2 + x_1^2 - 2x_2^2
\]
has only one stationary point, and that it is neither a maximum nor a minimum,
but a saddle point. Sketch the contour lines of $f$.

\paragraph{Exercise 1.3} % Nocedal 2.7
Suppose that $f$ is a convex function. Show that the set of global minimizer
of $f$ is a convex set.

\paragraph{Exercise 1.4} % Nocedal 2.8
Consider the function
\[
  f(x_1, x_2) = (x_1+x_2^2)^2.
\]
At the point $x=(1,0)^\tp$ we consider the search direction $p=(-1,1)^\tp$.
Show that $p$ is a descent direction and find all minimizers of the problem.

\paragraph{Programming 1}  % Nocedal 3.1
Program
\begin{itemize}
  \item the steepest descent method and
  \item Newton's method
\end{itemize}
with the backtracking line search.

Use them to minimize
\begin{itemize}
  \item the Rosenbrock function (\ref{eq:rosenbrock}),
  \item the \emph{sphere function}
    \[
      f(x) = \sum_{i=1}^n x_i^2,
    \]
    and
  \item one function of your choice from
    \url{https://en.wikipedia.org/wiki/Test_functions_for_optimization}.
\end{itemize}
Set the initial step length $\alpha_0=1$ and print the step length used at
each method at each iteration. First try the initial point $x_0=(1.2,
1.2)^\tp$, and then the more difficult point $x_0=(-1.2, 1)^\tp$.

% TODO add Hintermueller exercises?

\end{document}
% =============================================================================
